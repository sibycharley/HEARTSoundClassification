\documentclass[12pt]{article}

\title{Heart Sound Classification - High Level Design Document}
\author{Satendra Singh, Siby Charley}

\begin{document}

\begin{titlepage}
\maketitle
\end{titlepage}

\tableofcontents

\pagebreak

\section{Introduction}
The High Level Design document aims to gives to brief idea on well defined problem statement, software strategy, evaluation mechanism, data sources etc

\section{Problem Statement}
The Human heart generates a lub dub sound under its normal operating conditions. But it give rises to many variations of the normal sounds when recorded from unhealthy subjects. Hence with an increase in the number of smart devices capable of listening to heart sounds, there is a need to conceptualize and prototype algorithms capable of automatically detecting normal v/s abnormal heart sounds.   

\section{Physionet Challenge 2016}
Physionet is basically an online resource platform offering complex physiologic signals and other related open source software. Its managed by members of MIT's Lab of Lab of Computational Physiology. In the year 2016, they launched a challenge aiming to create an automated way to distinguish between normal/abnormal heart sounds utilizing machine learning techniques.

\subsection{Challenge Data}
As part of the challenge, the physionet have put forward a training dataset of consisting of almost 3K heart sound wave files ranging from 5sec to 120 sec in duration. This is an unbalanced dataset with more number of Normal sounds compared to abnormal sounds. They have provided a validation set of data consisting of the 20% of the original test set. The full test set remains hidden and with the actaul challenge closed as of today, we might have to convert the validation set into the test set.   

\section{Software Strategy}
As part of the project implementation, we have broadly divided the efforts into two halves as Feature Engineering and Data modeling

\subsection{Feature Engineering}
The data is presented in the form of waveform(.wav format) and hence the feature engineering and extraction becomes a crucial pre-processing step. The clinical method of normal vs abnormal classifications are done in a acoustic manner by the doctor. Hence we are planning to extract acoustically motivated features like MFCC, STFT, CQT etc.

\subsection{Data modeling}
Once the acoustic features are extracted for every waveform in the dataset, the next step is to apply classification algorithms on top the data. Since this is a binary classification problem, we are giving preference to usage of SVM, Random Forest, Gradient boosting trees and their ensemble combinations

\section{Evaluation}
As per the original challenge, the evaluation is carried out by measuring the sensitivity, specificity and the overall score. the mathematical relation to find these are as below

\begin{equation}
Sensitivity, Se = wa1(Aa1/(Aa1 + Aq1 + An1)) + wa2((Aa2 + Aq2)/(Aa2 + Aq2 + An2))
\end{equation}

\begin{equation}
Specificty, Sp = wn1(Nn1/(Na1 + Nq1 + Nn1)) + wn2((Nn2 + Nq2)/(Na2 + Nq2 + Nn2))
\end{equation}

\begin{equation}
Overall Score = (Se + Sp) / 2
\end{equation}\\

\noindent
Nn1 = Normal heat sound in Normal clean record\\
Ng1 = Uncertain heat sound in Normal clean record\\
Na1 = Abnormal heat sound in Normal clean record\\
Nn2 = Normal heat sound in Normal noisy record\\
Ng2 = Uncertain heat sound in Normal noisy record\\
Na2 = Abnormal heat sound in Normal noisy record\\
\\
An1 = Normal heat sound in Abnormal clean record\\
Ag1 = Uncertain heat sound in Abnormal clean record\\
Aa1 = Abnormal heat sound in Abnormal clean record\\
An2 = Normal heat sound in Abnormal noisy record\\
Ag2 = Uncertain heat sound in Abnormal noisy record\\
Aa2 = Abnormal heat sound in Abnormal noisy record\\
\\	
wa1 = clean abnormal records/total abnormal record\\
wa2 = noisy abnormal records/total abnormal record\\
wn1 = clean normal records/total normal record\\
wn2 = noisy normal records/total normal record\\


\end{document}